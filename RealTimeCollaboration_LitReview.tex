\documentclass[prodmode,acmtecs]{acmsmall}
\usepackage[ruled]{algorithm2e}
\usepackage{hyperref}
\renewcommand{\algorithmcfname}{ALGORITHM}
\SetAlFnt{\small}
\SetAlCapFnt{\small}
\SetAlCapNameFnt{\small}
\SetAlCapHSkip{0pt}
\IncMargin{-\parindent}



\begin{document}
\markboth{B D Ball}{Technologies for Real-Time Collaboration in BioInformatics with Synchronized Web Browsers}

\title{Technologies for Real-Time Collaboration using Synchronized Web Browsers in BioInformatics} % title
\author{Brendan D Ball
\affil{University of Cape Town}}
\date{April 2015}

\begin{abstract}
This is an abstract
\end{abstract}

\keywords{WebRTC, WebSocket, Real-time collaboration, Bioinformatics}
\maketitle
\section{Introduction}

Bioinformatics has become a fast growing field both as a result of advancements in technology and advancements in experimental biological studies. This is evident when looking at the Human Genome Project. These advancements have created a need for collaboration amongst scientists, often working together from across the globe and often across disciplines \cite{li2013ngs,lee2007facilitating,tsiliki2014datamining}. One solution to this problem has been the SEQanswers project. This project tries to improve knowledge sharing by means of an open forum where scientists are able to have discussions specific to genomics \cite{li2012seqanswers}. Another successful community-driven project is the NGS WikiBook, which as the name implies, is a wiki containing information on next-generation sequencing and is a big contributor to knowledge sharing in the genomics field. Other platforms exist but are often closed communities and don't contribute as much to knowledge sharing \cite{li2013ngs}.\\

Knowledge sharing is just one aspect to collaboration. Other forms of collaboration are also necessary, specifically when researchers are in the process of analysing the results of their experiments, such as sequencing data. There is often a need for synchronized collaboration where researchers discuss their findings while viewing the results. Researchers often use visualisations of the data to analyse it which quickly becomes an unfavourable task when researchers are in different geo-locations. A project called DICODE has been designed for the biomedical field in an effort to solve this synchronous collaboration problem. This project incorporates a discussion view, mind-map view, and formal view (predefined knowledge items) allowing teams to collaborate on a project in different ways \cite{karacapilidis2011facilitating}. To create a fully collaborative environment there needs to be support for communication, data sharing, and coordination all relying heavily on technical infrastructure \cite{lee2007facilitating}.\\

This literature review focuses on web technologies which enable collaboration such as communication and data sharing, and specifically synchronized viewing across web browsers. Synchronizing web applications across different web browsers has become a popular feature for enabling collaboration. These web applications are often referred to as groupware. A popular example is Google Docs which is an online text editor. Google Docs enables multiple users to share a single document and are consequently able to view and edit that single document simultaneously. The document synchronizes edits across each user's web browser in real-time \cite{koren2013shared}. More frameworks and libraries are being developed to simplify the integration of synchronization features into any web application \cite{ozono2012real}. The main problem areas with implementing this synchronization are conflict resolution (if two or more users edit the same area in a document the edits somehow need to be merged) and response times (how long it takes for changes to propagate to other connected users) \cite{heinrich2012enriching}. Visualisations can also be synchronized across web browsers, for example viewing a 3D model while moving it around to see it from different angles \cite{marion2012real}. The reason for this literature review is researching the feasibility of implementing synchronization of 2D visualisations, more specifically synchronizing a genome browser to help bioinformaticians analyse a genomic sequence collaboratively. In the background older technologies such as AJAX, Comet, and plugins are briefly described after which the discussion follows, detailing WebSocket and WebRTC which are the two technologies being considered.  

\section{Background}
Before we look at new technologies which are most commonly used in groupware, we will look at some older technologies which were used and what their shortcomings are when it comes to implementing groupware. Synchronization requires a bidirectional data stream so that if a user (you) makes a change then that change can be pushed to the other users and if another user makes a change then that change can be pushed to you \cite{linner2012instant}.

\subsection{AJAX} 
AJAX stands for asynchronous JavaScript and XML. This technology allows requesting data from a web server using Javascript without reloading the webpage. It is widely used in websites today and it works well. The problem with AJAX is that it is not bidirectional, meaning the web server cannot push data to the client. The client always has to send a request to the web server first and then the web server subsequently responds by sending the data. This limitation can be overcome by polling the web server every interval to receive updated data. Every time a poll is made, a new connection is created with the web server. This is however very inefficient both in terms of the web server's resources and synchronization response times \cite{gutwin2011real}. 

\subsection{Comet}
Comet extends AJAX to overcome its limitation of not being bidirectional. It essentially relies on exploiting AJAX to do things it was never made for. Comet consists of three different hacks which allow bidirectional data streams. \\
\subsubsection{Long polling} sends a request to the web server using AJAX, but instead of receiving a response immediately, the request is only completed when the web server has new data to push to the client. This means the connection between the client and server may be kept open for much longer than a normal AJAX request would. After the client receives data that connection is closed and a new request is initiated. When data is being rapidly updated this becomes as inefficient as normal AJAX would've been.\\
\subsubsection{XHR Multipart Streaming} also uses AJAX but exploits a content type called multipart essentially allowing a connection to stay open between multiple messages. Multipart officially exists to allow large files to be sent in chunks. This often uses up much more memory on the client's device than necessary as it keeps the chunks in memory until the connection is closed. This can however be rectified by periodically closing the connection and creating a new request.\\
\subsubsection{XHR Iframe Streaming} makes use of a hidden iframe which connects to a web server and streams Javascript dynamically. The iframe is then processed to extract the data from the Javascript. This technique also results in memory problems just like multipart streaming above \cite{gutwin2011real}.\\
These exploits were used when there was no alternative besides using plugins and probably resulted in many unforeseen bugs.

\subsection{Plugins}
The final alternative to using AJAX or Comet used to be plugins. Plugins are essentially embedded applications into a web page. These applications were written as either Java applets, Flash plugins or Silverlight plugins. This allowed developers to access some native operating system APIs which weren't supported by web browsers themselves. In particular, plugins are able to create TCP socket connections with servers, which allowed access to a bidirectional data stream. This was probably the best solution as it allowed low latency persistent connections without using exploits like Comet. Plugins had other problems though, the first of which being that a client needs to install the plugin before being able to use the web application. Security was also a major problem with plugins, so much so that Apple refused to include Flash support in their iOS browser \cite{gutwin2011real}.\\\\

Gutwin and Lippold did an excellent job researching and comparing these technologies using experimental methods. They also go on to compare these technologies with WebSocket and recommend WebSocket to be used rather than Comet or AJAX due to it's better performance \cite{gutwin2011real}. Plugins have also become infeasible due to security issues and lack of support. \cite{wenzel2013towards}


\section{Discussion}
Since new web standards have been implemented, namely WebSocket and WebRTC, we no longer have to use sub par solutions to our problem. WebSocket and WebRTC are still new technologies but most updated web browsers support them by now. There are numerous papers exploring the possibilities and performance of WebSocket, as well as papers designing communication architectures on top of WebSocket \cite{panagiotakis2013architecture}.

\subsection{WebSocket}
WebSocket is designed to be the TCP socket of the web. Although Http and WebSocket do run on TCP, it is not natively accessible through web browsers. WebSocket was designed to fix this shortcoming and has many advantages over using a TCP socket through a plugin. \\

WebSocket is a full-duplex communication channel, which means that it allows bidirectional communication asynchronously \cite{rakhundereal}. WebSocket is built on top of existing web infrastructure, which results in both an advantage and a disadvantage \cite{skvorc2014performance}. The advantage being that it is able to handle proxy and firewall traversal which simplifies many problems that plain TCP connections would have given, given that most companies have proxies and firewalls installed \cite{almasi2013evaluation}. The disadvantage is that it is purely client-server and provides no support for peer-to-peer connections, which is a direct result of it being implemented on existing web infrastructure. This will become important when discussing WebRTC later on. \\

\subsubsection{WebSocket protocol}
We've already established that WebSocket is built on top of TCP. this means that when a client attempts to initiate a WebSocket connection with the server, it first needs to establish a TCP connection. A successful TCP handshake takes three messages. After the TCP handshake is completed, it takes another two messages for a successful WebSocket handshake after which the persistent connection is established and data can be sent back and forth. This protocol is visualized in \textit{figure 1} \cite{skvorc2014performance}.


\begin{figure}
\centerline{\includegraphics[scale=0.5]{websockethandshake}}
\caption{WebSocket Handshake.}
\label{figure:one}
\end{figure}

\subsubsection{Scalability}
When a user requests a web page an http request is sent, a connection is established, the relevant data is transmitted and the connection is closed. This means that http connections were designed to be short lived. This architectural feature benefits the scalability of traditional web applications. WebSocket connections are however persistent and not designed to be short lived. This has a big impact on server resources and makes it more difficult to scale web applications, which use WebSocket, to a large number of users. One possible solution to this is to use distributed clouds on which to host the web servers to improve network traffic and decrease resource utilization for each web server \cite{solomon2012distributed}. 



\subsection{WebRTC}
WebRTC is a new web technology built on top of RTP (Real-time transport protocol). It's primary goal is to allow real-time audio and video transmission between two browsers without any server relay in between. No plugins are required to use WebRTC as it has been added as a native API to most web browsers. Since it's growth in popularity it has been extended with a data channel as well which allows developers to transmit any arbitrary data \cite{karadogan2014evaluating}.\\

WebRTC allows a full-duplex communication channel similar to WebSocket. WebRTC is however fundamentally different to WebSocket. WebRTC uses UDP where WebSocket uses TCP. This means that by default WebRTC does not guarantee lossless, ordered data. The advantage of this is that UDP allows lower latency which is better for real-time communication when dealing with audio or video as lossless transmission is not essential. When dealing with a data channel lossless transmission may or may not be essential depending on the use case and WebRTC allows application layer reliable transmission should the developer require it. The last big difference between WebRTC and WebSocket is that WebRTC is designed to support peer-to-peer (P2P) communication directly between two browsers where WebSocket is entirely client-server \cite{karadogan2014evaluating}. Vogt, Werner, and Schmidt researched how to take advantage of WebRTC to create a browser based P2P content sharing platform and they were enthusiastic to explore this avenue further \cite{vogt2013leveraging}.

\subsubsection{Signalling}
P2P connections aren't easy to set up due to complications introduced by NAT (Network Address Translation). The solution to this problem is signalling. Signalling requires a third party, normally a web server to relay connection information between the two clients allowing them to set up a P2P connection. Once the connection is set up, the third party is no longer required and the two clients can communicate while the connection stays open. \textit{figure 2} demonstrates this interaction between the two clients and the third party \cite{karadogan2014evaluating}. WebRTC requires signalling but imposes no protocol on how it should be done. It is left to the system architects to decide how to implement signalling. There are two common methods used. The first method simply uses AJAX to communicate the necessary signalling information and the second method sets up a WebSocket connection and uses the Session Initiation Protocol (SIP) \cite{adeyeye2013determining}.

\begin{figure}
\centerline{\includegraphics[scale=0.5]{webrtcsignalling}}
\caption{WebSocket Handshake.}
\label{figure:two}
\end{figure}

\subsubsection{Scalability}
When purely looking at P2P networks they are inherently scalable. However real-time collaboration may very likely exist between more than two users. When trying to synchronize more than two browsers using P2P connections it quickly becomes complex. There are different architectures for solving this problem, the most obvious of which being a full mesh architecture. In a group of size ${N}$ where the whole group aims to be synchronized, a full mesh requires every peer to set up a connection with every other peer in the group. This results in $N^2$ connections which quickly becomes infeasible as $N$ grows. An alternative to a full mesh is a star architecture where one client acts as a server in the group and relays data between all clients. This could pose a problem should the main client disconnect but can be solved by implementing a framework that automatically fails over to another client. The last architecture is a multipoint control unit (MCU) which essentially is a client-server architecture where a single server relays data between clients. This architecture would behave similar to a WebSocket architecture \cite{karadogan2014evaluating}.



\section{Conclusions}

sd
\appendix
\bibliographystyle{ACM-Reference-Format-Journals}
\bibliography{rtc}

\end{document}

