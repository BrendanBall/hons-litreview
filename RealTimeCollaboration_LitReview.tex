\documentclass[prodmode,acmtecs]{acmsmall}
\usepackage[ruled]{algorithm2e}
\renewcommand{\algorithmcfname}{ALGORITHM}
\SetAlFnt{\small}
\SetAlCapFnt{\small}
\SetAlCapNameFnt{\small}
\SetAlCapHSkip{0pt}
\IncMargin{-\parindent}



\begin{document}
\markboth{B D Ball}{Technologies for Real-Time Collaboration in BioInformatics with Synchronized Web Browsers}

\title{Technologies for Real-Time Collaboration using Synchronized Web Browsers in BioInformatics} % title
\author{Brendan D Ball
\affil{University of Cape Town}}
\date{April 2015}

\begin{abstract}
This is an abstract
\end{abstract}

\keywords{WebRTC, WebSocket, Real-time collaboration, Bioinformatics}
\maketitle
\section{Introduction}

Bioinformatics has become a fast growing field both as a result of advancements in technology and advancements in experimental biological studies. This is evident when looking at the Human Genome Project. These advancements have created a need for collaboration amongst scientists, often working together from across the globe and often across disciplines. One solution to this problem has been the SEQanswers project. This project tries to improve knowledge sharing by means of an open forum where scientists are able to have discussions specific to genomics \cite{li2012seqanswers}. Another successful community-driven project is the NGS WikiBook, which as the name implies, is a wiki containing lots of information on next-generation sequencing and is a big contributor to knowledge sharing in the genomics field \cite{li2013ngs}. Other platforms exist but are often closed communities and don't contribute as much to knowledge sharing.\\\\
Knowledge sharing is just one aspect to collaboration. Other forms of collaboration are also necessary, specifically when researchers are in the process of analysing the results of their experiments, such as sequencing data. There is often a need for synchronous collaboration where researchers discuss their findings while viewing the results. Researchers often use visualisations of the data to analyse it which quickly becomes an unfavourable task when researchers are in different geo-locations. A project called DICODE has been designed for the biomedical field in an effort to solve this synchronous collaboration problem. This project incorporates a discussion view, mind-map view, and formal view (predefined knowledge items) allowing teams to collaborate on a project in different ways.\cite{karacapilidis2011facilitating}.

\section{Background}
df
\section{Discussion}
df
\section{Conclusions}
sd
\appendix
\bibliographystyle{ACM-Reference-Format-Journals}
\bibliography{rtc}

\end{document}

